%----------------------------------------------------------
%GitHub version
% Filename: NN-Inequalities-2019.tex
% submitted to
%----------------------------------------------------------


 \documentclass[12pt]{article}

 \usepackage{epsfig}       %for Postscript files
 \usepackage{subfigure}    %for parallel figures
 \usepackage{latexsym}
 %\usepackage{dsfont}
 \usepackage{graphicx}
 \usepackage{amsfonts}
 \usepackage{mathrsfs}
 \usepackage{amsfonts}
 %\usepackage{extarrows}
 \usepackage{amssymb,amsthm,amsmath,amscd,fancybox,graphicx}
 \usepackage{amssymb,amsmath,euscript}
 %\usepackage[numbers,sort&compress]{natbib}
 %\usepackage{epstopdf}
 \usepackage{float}
% \usepackage{refcheck}
% \usepackage{datetime}
 \usepackage{color}
 \usepackage{rotating,enumerate}
 \usepackage{tikz}
 \usetikzlibrary{positioning}


 \textheight=8.75in
 \textwidth=6.25in
 \topmargin=-.5in
 \oddsidemargin=0.0in
 \evensidemargin=0.0in
 \renewcommand{\baselinestretch}{1.05}
% \renewcommand{\Re}{{\rm I}\!  {\rm R}}
% \renewcommand{\Cx}{{\rm I}\!  {\rm C}}

 \renewcommand{\Re}{{\rm I}\! {\rm R}}
 \newcommand{\Cx}{\mathds{C}}
 \newcommand{\KK}{\mathds{K}}
 \newcommand{\rn}{\Re^n}
 \newcommand{\rl}{\Re^l}
 \newcommand{\s}{{\cal S}}
 \newcommand{\K}{{\cal K}}
 \renewcommand{\H}{{\cal H}}
 \newcommand{\MV}{\mathbb{V}}
 \newcommand{\MW}{\mathbb{W}}
 \newcommand{\ds}{\displaystyle}
 \def \gret {\succeq_{_{{\cal K}^n}} }
 \def \gre {\succ_{_{{\cal K}^n}} }
 \def \less {\preceq_{_{{\cal K}^n}} }
 \def \les {\prec_{_{{\cal K}^n}} }
 \def \rla{\rangle}
 \def \lla{\langle}
 \def \eps {\epsilon}
 \def \veps {\varepsilon}
 \def \t{\theta}
 \def \sgn {{\rm sgn}}
 \def \beginproof{\par\noindent {\bf Proof.}\ \ }
 \def \endproof{\hskip .5cm $\Box$ \vskip .5cm}
 \def\reff#1{{\rm(\ref{#1})}}

 \newcommand{\IK}{{\rm I}\!  {\rm K}}
 \newcommand{\intK}{{\rm int} \IK}
 \newcommand{\phif}{ \phi_{_{\rm F}} }
 \newcommand{\phil}{ \phi_{_{\rm L}} }
 \newcommand{\phifb}{ \phi_{_{\rm FB}} }
 \newcommand{\psifb}{ \psi_{_{\rm FB}} }
 \newcommand{\Phifb}{ \Phi_{_{\rm FB}} }
 \newcommand{\Psifb}{ \Psi_{_{\rm FB}} }
 \newcommand{\phinr}{ \phi_{_{\rm NR}} }
 \newcommand{\psinr}{ \psi_{_{\rm NR}} }
 \newcommand{\Psinr}{ \Psi_{_{\rm NR}} }
 \newcommand{\phifbp}{ \phi_{_{\rm FB}}^p }
 \newcommand{\phifdb}{ \phi_{_{\rm D-FB}}}
 \newcommand{\phifdbp}{ \phi_{_{\rm D-FB}}^p }
 \newcommand{\psifdbp}{ \psi_{_{\rm D-FB}}^p }
 \newcommand{\psifbp}{ \psi_{_{\rm FB}}^p }
 \newcommand{\phinrp}{ \phi_{_{\rm NR}}^p }
 \newcommand{\psinrp}{ \psi_{_{\rm NR}}^p }
 \newcommand{\psip}{ \psi_{p} }
 \newcommand{\phip}{ \phi_{p} }
 \newcommand{\Psip}{ \Psi_{p} }
 \newcommand{\Phip}{ \Phi_{p} }
 \newcommand{\phialpha}{ \phi_{\alpha, p} }
 \newcommand{\Phialpha}{ \Phi_{\alpha, p} }
 \newcommand{\psialpha}{ \psi_{\alpha, p} }
 \newcommand{\Psialpha}{ \Psi_{\alpha, p} }
 \newcommand{\phitheta}{ \phi_{\theta, p} }
 \newcommand{\Psitheta}{ \Psi_{\theta, p} }


%--------------------------------------- begin document ---------------------------------------
 \begin{document}

 \newtheorem{property}{Property}[section]
 \newtheorem{proposition}{Proposition}[section]
 \newtheorem{append}{Appendix}[section]
 \newtheorem{definition}{Definition}[section]
 \newtheorem{lemma}{Lemma}[section]
 \newtheorem{theorem}{Theorem}[section]
 \newtheorem{example}{Example}[section]
 \newtheorem{corollary}{Corollary}[section]
 \newtheorem{condition}{Condition}
 \newtheorem{remark}{Remark}[section]
 \newtheorem{assumption}{Assumption}[section]
 \newtheorem{algorithm}{Algorithm}[section]
 \newtheorem{problem}{Problem}[section]


 \medskip

 \begin{center}
 {\large \bf Neural network for solving system of inequalities
 based on systematically generated smoothing functions }

 \vskip1cm


 B. Saheya
 \footnote{E-mail: saheya@imnu.edu.cn. The author's work is supported by National
 Key R\&D Program of China (Award Number: 2017YFC1405605) and
Natural science foundation of Inner Mongolia ( Award Number: 2019LH01001).} \\
 College of Mathematical Science \\
 Inner Mongolia Normal University \\
 Hohhot 010022, Inner Mongolia, P. R. China.  \\


 \vskip1cm


% Tao Zhang
% \footnote{E-mail: taozhang@imnu.edu.cn}\\
% College of Mathematical Science \\
% Inner Mongolia Normal University \\
% Hohhot 010022, Inner Mongolia, P. R. China.  \\
%
%
% \vskip 1cm
%
%
 Jein-Shan Chen
 %\footnote{Corresponding author. E-mail:jschen@math.ntnu.edu.tw. The author's work is supported by
 % Ministry of Science and Technology, Taiwan.}
 \\
 Department of Mathematics  \\
 National Taiwan Normal University \\
 Taipei 11677, Taiwan.

 \vskip1cm

 Guoqing Chen
 \footnote{Corresponding author. E-mail:cgq@imu.edu.cn. } \\
 College of Mathematical Science \\
 Inner Mongolia University \\
 Hohhot 010021, Inner Mongolia, P. R. China.  \\

 \medskip

 September 21, 2019

 \end{center}


 \medskip

 \noindent
 {\bf Abstract} \
 In this paper, we summarize several systematic ways of constructing smoothing
 functions and illustrate 11 smoothing functions accordingly for plus function
 which  include some new smoothing functions. Then, based on these systematically
 generated  smoothing functions, a unified neural network model is proposed for
 solving  system of inequalities. The issues regarding the equilibrium point, the
 trajectory, and the stability properties of the  neural network are addressed.
 Moreover, numerical experiments with comparison are presented, which suggests
 what kind of smoothing functions work well along  with the neural network approach.


 \vspace{0.8cm}

 \noindent
 {\bf Keywords.}\  System of inequalities, Neural network, Plus function, Smoothing method.

 \medskip


%---------------------------------------------------------------------------------------------- Section 1
 \section{Introduction} \label{sec1}


 In this paper, we are interested in finding a solution to system of inequalities
  by neural network approach. The mathematical format of the system of inequalities is given by
 \begin{equation} \label{INEQUALITY}
 f(x) \leq 0,
 \end{equation}
 where $f(x)=( f_1(x),f_2(x),\cdots, f_n(x))^\mathrm{T}$,
 $ x \in \Re^{n }$ and $f_i:\Re^{n } \rightarrow \Re, (i=1,2,\cdots,n)$ are
 continuous differentiable mapping.
 The system of inequalities have been paid much attention due to its many applications,
 including data analysis, computer-aided design problems, image reconstructions etc.
   \\

The main idea of a class of important methods for solving inequalities \eqref{INEQUALITY} is to
transform them into the following nondifferentiable equations via projection function
\begin{equation}\label{plusf}
  [f (x)]_+ = 0.
\end{equation}
Where $(x)_+ = (\max\{0,x_1\},\cdots, \max\{0, x_n\})^\mathrm{T}, x \in \rn$.

 Iterative methods for nondifferentiable equations have been studied for decades, among these methods
 smoothing methods and semismooth methods for nondifferentiable equations  have been studied extensively in the last few
years. For the above nondifferentiable equation with  projection operator \eqref{plusf}, some smoothing functions have been proposed by scholars.  Huang \cite{HZ2008} gives a smoothing type algorithm to solving  \eqref{plusf}, a smoothing Newton method was designed by Fan \cite{fy18}, based on a new family smoothing function a nonmonotone Newton method was proposed by Chen \cite{Chen2016} for equalities and inequalities.

 Basically, the main ideas for the aforementioned approaches can be categorized
 as below.  The first one  is  to recast the system of inequalities (\ref{INEQUALITY}) as a system of nonsmooth equations
 and then tackle with the nonsmooth equations by using nonsmooth Newton  algorithm \cite{Qi93} or
 other subgradient approaches. The Second
 one is transform it to smoothing equations and applying smoothing Newton algorithm \cite{QSZ00}.
 In this paper,  neural network method is used to solve the system of inequalities (\ref{INEQUALITY}),
  which is not found in other literatures.
 \\

 Now, we elaborate more about the idea of neural network approach. For many
 practical applications, the real-time solutions of optimization problems are
 mostly required. One possible and very promising approach to solving the real-time
 optimization problem is to apply artificial neural networks. The neural networks
 can be implemented using integrated circuits. Neural networks for optimization
 were first introduced in the 1980s by Hopfield and Tank \cite{HT85, TH86}. Since
 then, significant research results have been achieved for various optimization
 problems, including linear programming \cite{ZUH95}, quadratic programming
 \cite{BP93}, linear complementarity problems \cite{LQ99}, nonlinear complementarity
 problem \cite{LQQ01} and nonlinear programming \cite{CU93}. In general, the
 essence of neural network approach is to construct a nonnegative energy function
 and establish a dynamic system that represents an artificial neural network.
 A first order differential equation represents the dynamic system. Furthermore,
 it is expected that the dynamic system will converge to its static state (or an
 equilibrium point), which corresponds to the solution for the underlying
 optimization problem, starting from an initial point.
 \\

 In this paper, we present systematical
 ways about how to construct smoothing functions for system of inequalities \eqref{INEQUALITY}
 and illustrate 11 smoothing functions accordingly. After that, we design a gradient descent
 neural network model by using these 11 different smoothing functions. We not
 only discuss the stability of the neural networks, but also give numerical comparison
 for these smoothing functions.  The innovation of this paper is twofold.
 Firstly, some new smoothing functions of projection functions are constructed by summing up
 the smoothing methods. Secondly, the nonlinear system of inequalities is solved by neural network
 approach. The third is  numerical comparison, which suggest what kind of smoothing functions work well
 along with the neural network approach for system of  inequalities.
 \\




%---------------------------------------------------------------------------------------------- Section 2
 \section{Preliminaries} \label{sec2}



 By looking into the mathematical format of the \eqref{plusf}, it is clear that the plus
  function $(x)_+$ is the key component. Indeed, the plus function
 also plays an important role in a lot of applications, like machine learning and
 image processing, etc. In particular, the plus
  function $(x)_+$ is not
 differentiable at $x=0$, which causes limits in analysis and application. To conquer
 this hurdle, researchers consider smoothing methods and construct smoothing functions
 for it.
 The definition of smoothing function of plus function $(x)_+$ is given by following definition.
%----------------------------------------------------------------------------------- Definition 2.1
 \begin{definition}
 If  $\phi(\mu, \cdot):\Re \to \Re$ with respect to
 $(\alpha)_+$ satisfying
 \begin{equation*} \label{condition}
 \lim_{\mu \downarrow 0} \phi(\mu,\alpha) =(\alpha)_{+},
 \end{equation*}
 for all $\alpha\in \Re$. Then call $\phi(\mu, \cdot)$ a smoothing function of plus function $ (x)_+$.
 \end{definition}

 The smoothing technique has been exploited to solving problems in
  machine learning, statistics, cone programming. These successful cases motivate a systematic
  study of the smoothing approach. How to generate new smoothing functions?
  What is a common property of smoothing functions of plus function ?
  We summarize all possible ways to construct smoothing functions for plus
  function $(x)_+$
 as below. For more details, please refer to \cite{BM12, CM96, KR92, Nesterov05, QD02, VOY15}.
 \\

 \noindent
 {\bf (1) Smoothing by the convex conjugate}

 \medskip

 Let $X$ be a real topological vector space, and let $X^{*}$ be the dual
 space to $X$. For any function $f: {\rm dom} f  \to \Re$, its convex conjugate
 $f^*: ({\rm dom}f)^* \to \Re$ is defined in terms of the supremum by
 \[
 f^*(y) := \sup_{x \in {\rm dom} f} \left \{x^T y - f(x) \right \}.
 \]
 In light of this, one can build up smooth approximation of $f$, denoted by $f_{\mu}$,
 by adding strongly convex component to its dual $g:=f^*$, namely,
 \[
 f_{\mu}(x) = \sup_{z\in {\rm dom} g} \left \{z^T x - g(x) - \mu d(z) \right \}
 = (g+\mu d)^*(x),
 \]
 for some $1$-strongly convex and continuous function $d(\cdot)$ (called proximity
 function). Here, $d(\cdot)$ is $1$-strongly convex which satisfies
 \[
 d( (1-t)x + t y) \leq (1-t) d(x) + t d(y) - \frac{1}{2} t(1-t) \|x-y\|^2,
 \]
 for all $x, y$ and $t \in (0,1)$. Note that
 \(
 (x)_+ = \sup_{0 \leq z \leq 1 }zx.
 \)
 If we take $d(z):=z^2/2$, then the constructed smoothing function via conjugation
 leads to
 \begin{equation}\label{phi1}
 \phi_1(\mu, x)=\sup_{0 \leq z \leq 1}\left\{ zx-\frac{\mu}{2}z^2 \right \}
 = \left\{
 \begin{array}{cl}
  x -\ds \frac{\mu}{2},  &    x  \geq \mu\\
 \frac{x^2}{2\mu},       & {\rm if} \ \  0 \leq x  \leq \mu, \\
  0,                     &    x <0. \\
 \end{array}
 \right.
 \end{equation}
 Which is the Pinar-Zenios function \cite{PZ1995}.

 Alternatively, choosing $d(y):=1-\sqrt{1-y^2}$
 gives another smoothing function:
 \begin{equation}\label{phi2}
 \phi_2(\mu, x) =\left\{
    \begin{array}{cl}
            \sqrt{x^2+\mu^2}-\mu, &    x  > 0\\
            0,                   &    x \leq 0.
    \end{array}
    \right.
 \end{equation}
\\

 \noindent
 {\bf (2) The Moreau proximal smoothing}

 \medskip

 Suppose that $\mathbb{E}$ is an Euclidean space and $f:\mathbb{E} \to (-\infty,\infty]$
 is a closed and proper convex function. One natural tool for generating an approximate
 smoothing function is through the use of the so-called proximal map introduced by Moreau
 \cite{Mor65}. The Moreau proximal approximation yields a family of approximations
 $\{ f^{\rm px}_{\mu}\}_{\mu >0}$ as below:
 \begin{equation*}\label{moreau}
 f^{\rm px}_{\mu}(x)=\inf_{u\in \mathbb{E}} \left\{ f(u)+\frac{1}{2\mu}\|u-x\|^2 \right\}.
 \end{equation*}
 It is known that the Moreau proximal approximation $f^{\rm px}_{\mu}(x)$ is convex
 continuous, finite-valued, and differentiable with gradient $\nabla f^{\rm px}_{\mu}$
 which is Lipschitz continuous with constant $\frac{1}{\mu}$, see \cite{Mor65}.
 When applying the Moreau proximal smoothing way \cite{BM12} to construct the smoothing function
 for the plus function $(x)_+$, we can obtain the Pinar-Zenios function\cite{PZ1995}  $\phi_1(\mu, x)$
 defined by \eqref{phi1}.



 \noindent
 {\bf (3) Nesterov's smoothing}

 \medskip

 There is a class of nonsmooth convex functions considered in \cite{Nesterov05},
 which is given by
 \[
 q(x)=\max\{ \lla u,Ax \rla - \phi(u) \, | \, u \in Q\}, \quad x \in \mathbb{E},
 \]
 where $\mathbb{E}$, $V$ are finite dimensional vector spaces, $Q \subseteq V^*$
 is compact and convex, $\phi$ is a continuous convex function on $Q$, and
 $A :\mathbb{E} \to V $ is a linear map.
 The smooth approximation of $q$ suggested in \cite{Nesterov05} is described by
 the convex function
 \begin{equation*}\label{nesterov}
 q_\mu(x) =\max \{ \lla u, Ax \rla -\phi(u)-\mu d(u) \, | \, u\in Q\},
 \quad x \in \mathbb{E},
 \end{equation*}
 where $d(\cdot)$ is a prox-function for $Q$. It was proved in
 \cite[Theorem 1]{Nesterov05} that the convex function $q_\mu(x)$ is $C^{1,1}(\mathbb{E})$.
 More specifically, its gradient mapping is Lipschitz continuous with constant
 $L_\mu= \ds \frac{\|A\|^2}{\sigma \mu}$ and the gradient is described by
 $\nabla q_\mu(x)= A u_\mu(x)$, where $u_\mu(x)$ is the unique minimizer of \eqref{nesterov}.
 \\

 For the absolute value function $q(x) = (x)_+ $ with $x\in \Re^1$. Let $A=1$,
 $b=0$, $\mathbb{E}=\Re^1$, $Q=\{u\in \Re^1 \, | \, 0 \leq u \leq 1\}$ and taking
 $d(u):=\frac{1}{2}u^2$. Then, we have
 \begin{eqnarray*}
 q_\mu(x)
    &=& \max_{u \in Q} \left\{ \langle Ax-b,u \rangle - \mu d(u)\, | \ \right \}
    = \max_{u \in Q} \left\{ xu - \frac{\mu}{2}u^2 \right \}
    = \left\{
 \begin{array}{cl}
    x -\frac{\mu}{2}, &  {\rm if} \  x > \mu, \\
    \frac{x^2}{2\mu},  & {\rm if} \ 0 \leq x \leq \mu, \\
    0,                 &  {\rm if} \  x < 0. \\
  \end{array}
    \right.
 \end{eqnarray*}
 As we see, it also yields the Pinar-Zenios function \cite{PZ1995} $\phi_1(\mu, x)$ defined
 by \eqref{phi1}.
 \\



 \noindent
 {\bf (4) The infimal-convolution smoothing technique}

 \medskip

 Suppose that $\mathbb{E}$ is a finite vector space and
 $f, g:\mathbb{E} \to (-\infty,\infty]$. The infimal-convolution of $f$ and $g$,
 $f \, \Box \, g : \mathbb{E} \to [-\infty, +\infty]$ is defined by
 \[
 (f \, \Box \, g)(x) = \inf_{y \in \mathbb{E}} \left \{ f(y)+g(x-y) \right \}.
 \]
 In light of the concept of infimal convolution, one can also construct smoothing
 approximation functions. More specifically, we consider
 $f : \mathbb{E} \to (-\infty,\infty]$ which is a closed proper convex
 function and let $\omega : \mathbb{E} \to \mathbb{R} $ be a $C^{1,1}$
 convex function with Lipschitz gradient constant $1/\sigma$ ($\sigma> 0$).
 Suppose that for any $\mu >0$ and any $x \in \mathbb{E}$, the following
 infimal convolution is finite:
 \begin{equation*} \label{inf-cov}
 f^{\rm ic}_\mu(x)
 = \inf_{u\in \mathbb{E}} \left\{ f(u)+\mu \omega(\frac{x-u}{\mu}) \right \}
 = (f \, \Box \, \omega_\mu) (x),
 \end{equation*}
 where $\omega_\mu(\cdot)=\mu\omega(\frac{\cdot}{\mu})$.
 Then, $f^{\rm ic}_\mu$ is called the infimal-convolution $\mu$-smooth approximation
 of $f$. In particular, when $\mu\in \Re_{++}$ and $p\in (1,+\infty)$, the
 infimal convolution of a convex function and a power of the norm function is
 obtained as below:
 \begin{equation*}
 f \, \Box \, \left( \frac{1}{\mu p}\|\cdot\|^p \right)
 = \inf_{u\in \mathbb{E}} \left\{ f(u)+ (\frac{1}{\mu p}\|x-u\|^p) \right \}.
 \end{equation*}

 \noindent
 For the plus function, it can be verified that
 $f_\mu (x)=((\cdot)_+) \, \Box \, \left( \frac{1}{\mu *p}|\cdot|^p \right)$ is the
 smoothing function of order $p$, i.e.,
 \begin{equation}\label{phi-p}
 \phi^{(p)} (\mu, x)= f_\mu (x) = \left\{
 \begin{array}{cll}
 x-\frac{p-1}{p}\mu^{\frac{1}{p-1}}, & {\rm if} & x > \mu^{\frac{1}{p-1}}, \\
 \frac{x^p}{\mu p},                & {\rm if} & 0 \leq x \leq  \mu^{\frac{1}{p-1}},\\
  0,                                 & {\rm if} & x<0.
 \end{array}
 \right.
 \end{equation}

 Note that when $p = 2$ in the above expression \eqref{phi-p}, the smoothing function
 of order $p$ reduces to the Pinar-Zenios function $\phi_1(\mu, x)$ as shown in \eqref{phi1}.

 Figure \ref{p-order-smoothr} depicts the smoothing function of order $p$ with various value
 of $p$. It can be observed from the Figure \ref{p-order-smoothr} that the smoothing curve is
 further and further away from plus function as $p$ increases.
 To the contrast, plugging $p = 2$ into infimal-convolution formula
 yields the Moreau approximation \eqref{moreau}. For more details
 about infimal-convolution and its induces approximation functions, please refer to
 \cite{BM12, BC2016}.



 \begin{figure}
 \centering
 \includegraphics[scale=0.6]{p-oder-smooth-function.pdf}\\
 \caption{ $(x)_+$ and smoothing function of order $p$ ($\mu=0.3$).}\label{p-order-smoothr}
 \end{figure}



 \noindent
 {\bf (5) The convolution smoothing technique}

 \medskip

 The smoothing approximation via convolution for the plus function
 is a popular way, which can be found in \cite{CM96, KR92, QD02, VOY15}. Its
 construction idea is described as follows. First, one constructs a smoothing
 approximation for the plus function $(x)_+ = \max\{0,x\}$. To this end, we
 consider the piecewise continuous function $d(x)$ with finite number of pieces
 which is a density (kernel) function, that is, it satisfies
 \[
 d(x)\geq 0 \quad {\rm and} \quad \int_{-\infty}^{+\infty}d(x)dx = 1.
 \]
 Next, define $\hat{s}(\mu, x) := \frac{1}{\mu}d\left(\frac{x}{\mu}\right)$,
 where $\mu$ is a positive parameter. Suppose that
 $\int_{-\infty}^{+\infty}\left|x\right|d(x)dx < +\infty$, then a smoothing
 approximation (denoted by $\hat{p}(\mu, x)$) for $(x)_+$ is obtained as below:
 \begin{equation*}
 \hat{p}(\mu, x) = \int_{-\infty}^{+\infty}(x-s)_+\hat{s}(\mu, s)ds
 = \int_{-\infty}^{x}(x-s)\hat{s}(\mu, s)ds.
 \end{equation*}
 There are some well-known kernel functions

  \begin{eqnarray*}
   d_1(x) & =&
 \left\{
 \begin{array}{clc}
 1 & \rm{if} \; 0\leq x \leq 1,\\
 0 & \rm{otherwise}.
 \end{array}
 \right.
 d_2(x) = \frac{e^{-x}}{(1+e^{-x})^2},  \\
 d_3(x) & =&
 \left\{
 \begin{array}{clc}
 1 & \rm{if} \; -\frac{1}{2}\leq x \leq \frac{1}{2},\\
 0 & \rm{otherwise},
 \end{array}
 \right.
 d_4(x) = \frac{2}{(x^2 + 4)^{\frac{3}{2}}}, \\
 \end{eqnarray*}
  From  $d_1(x)$  we can also derive  Pinar-Zenios function $\phi_1$ defined by \eqref{phi1}.
 Based on kernel functions $d_2(x), d_3(x), d_4(x)$ , we obtain following  smoothing functions for the plus function \cite{CM96, QD02}:
 \begin{eqnarray}
  {\phi}_3(\mu, x) &=& x + \mu
 \log \begin{pmatrix}
 1 + e^{-\frac{x}{\mu}} \\
 \end{pmatrix}. \label{phi3} \\
 {\phi}_4(\mu, x) &=&
 \left\{
 \begin{array}{cll}
 x & {\rm if} & x \geq \frac{\mu}{2}, \\
 \frac{1}{2\mu}\left(x + \frac{\mu}{2}\right)^2 & {\rm if} & -\frac{\mu}{2} < x < \frac{\mu}{2}, \\
 0 & {\rm if} & x\leq -\frac{\mu}{2}.
 \end{array}
 \right. \label{phi4} \\
  {\phi}_5(\mu, x) &=& \frac{\sqrt{4\mu^2 + x^2} + x}{2}. \label{phi5}
 \end{eqnarray}
The function $\phi_3$  is called the neural networks smooth function \cite{CM96} ,
the function $\phi_4$  is called the Zang smooth function \cite{CM96,Zang1980} and
the function $\phi_5$  is called the Chen-Harker-Kanzow-Smale smooth function \cite{CM96, CH1993}.

\noindent
 {\bf (6) Smoothing by epigraph}

 \medskip

We consider the  maximum function
\[
 f(x) = \max_{i \in I}f_i(x),
\]
where $f_i(x)\in C^k(\rn)$. The maximum function $f$ is in general not differentiable.
One kind of approach is  replace the minimization of $f $ by means of the minimization
of the linear function $y$ on the epigraph of $f$. Therefore, the nonsmooth unconstrained
minimization problem is replaced by a smooth, but constrained minimization. A barrier
(interior approximation) for the $epi(f)$ constrained is constructed by
\[
\prod_{i\in I}\left ( y - f_i(x)\right ) \geq \mu, \ (\mu > 0),
\]
or, equivalently form
\[
\sum_{i\in I}\log \left ( y - f_i(x)\right ) \geq \log(\mu), \ (\mu > 0).
\]

From this we get a function $f_\mu$ whose graph lies in the interior of $epi(f)$
and which is defined by
\[
\sum_{i\in I}\log \left ( y - f_i(x)\right ) = \log(\mu).
\]
It is  shown that $f_\mu$  is a  $C^k$-function and have some good
property \cite{GuerraVazquez2001}.
For function $(x)_+ = \max(0,x)$, we get the smoothing function

\begin{equation*}
f_\mu = \frac{x}{2} + \sqrt{\mu + \frac{x^2}{4}}.
\end{equation*}
Which is the smoothing function $\phi_5(\mu,x)$ defined by \eqref{phi5}.


 \noindent
 {\bf (7) Indirect smoothing technique}

 \medskip
 On the other hand, we can smooth out the plus function by looking at
 other smoothing functions in relation to the plus function.  For
 example, we can use the relationship between the absolute value
 function and the plus function,i.e.,$(x)_+ = \frac{x + |x|}{2}$.
 In reference \cite{saheya2019}, eight  smoothing functions of
 absolute value functions are given. From this we can get eight
 smoothing functions for plus function. In addition to the
 functions or very similar functions that we have constructed
 in the previous sections, there are several new functions
 as following
 \begin{eqnarray}
 \phi_6(\mu, x) &=&  \frac{1}{2}(x + \mu \log \left( \cosh \left( \frac{x}{\mu} \right) \right)). \label{phi6}\\
 \phi_7(\mu, x) &=&  \frac{1}{2}(x + \mu \left [ \log \left( 1 + e^{-\frac{x}{\mu}} \right) +
                \log \left( 1 + e^\frac{x}{\mu} \right)  \right] ). \label{phi7}  \\
 \phi_8(\mu, x)
 &=& \left \{
 \begin{array}{cll}
  x& {\rm if} & x \geq \frac{\mu}{2}, \\
  \frac{1}{2}(x + \frac{x^2}{\mu} + \frac{\mu}{4}) & {\rm if} & -\frac{\mu}{2} < x < \frac{\mu}{2}, \\
  0 & {\rm if} & x \leq -\frac{\mu}{2}.
 \end{array}
 \right. \label{phi8}  \\
 \phi_9(\mu, x) &=&
 \left\{
 \begin{array}{cll}
  x& {\rm if} &  x > \mu, \\
  \frac{1}{2}(x -\frac{x^4}{8\mu^3} + \frac{3x^2}{4\mu} + \frac{3\mu}{8} )& {\rm if} &  -\mu \leq x \leq \mu, \\
  0 & {\rm if} & x < -\mu.
 \end{array}
 \right. \label{phi9}  \\
 \phi_{10}(\mu, x) &=&  \frac{1}{2}(x + x \, {\rm erf} \left(\frac{x}{\sqrt{2}\mu}\right)
                + \sqrt{\frac{2}{\pi}}\mu e^{-\frac{x^2}{2\mu^2}}). \label{phi10}
\end{eqnarray}


 \begin{figure}
 \centering
 \includegraphics[scale=0.6]{All-Smooth.pdf}\\
 \caption{The graphs of $(x)_+$ and the smoothing functions $\phi_i, i=1,\cdots,10$
 ($\mu=0.4$).}\label{All-Smooth}
 \end{figure}



 To sum up, we have 11 smoothing functions in total through the above
 constructions. Figure \ref{All-Smooth} depicts the graphs of all the
 aforementioned smoothing functions $\phi_i$, $i=1,\cdots,10$, $p$-order
 smoothing function $\phi^{(p)}, (p=1.5)$ and the plus function. Not only
 from the geometric view, $\phi_i$, $i=1,\cdots,10$ and $\phi^{(p)}$ are clearly smoothing
 functions of $(x)_+$, it can be also verified theoretically
 in Proposition \ref{Proposition 2.1}.
\medskip

%---------------------------------------------------------------------------------- Proposition 2.1
 \begin{proposition} \label{approximationBound}
 Let $\phi_i:\Re^2 \to \Re$ for $i=1,\ldots, 10$, $\phi^{(p)}$ be defined as in
 \eqref{phi1}--\eqref{phi10}, respectively and  $\mu >0$ .
 Then, we have
    \begin{description}
    \item[(i)]  $\phi_{i}$ is continuously differentiable at
                $(\mu, x) \in \Re_{++} \times \Re$;
    \item[(ii)] $\ds \lim_{\mu \downarrow 0} \phi_{i} (\mu, x)=(x)_+$.
    \item[(iii)] $ \frac{\partial \phi_i(\mu,t)}{t} \geq 0,  \forall \mu >0.$
%    \item[(iv)] If $t_1 \leq t_2$, then $\phi(\mu, t_1) \leq \phi(\mu, t_2)$.
    \item[(iv)]  If $\phi \in \{\phi_3,\phi_4, \phi_5, \phi_7, \phi_8, \phi_9, \phi_{10} \}$,
                then $\phi(\mu,t) > (x)_+$, and if $\phi \in \{\phi_1, \phi_2, \phi_6, \phi^{(p)}\}$,
                then $\phi(\mu,t) < (x)_+$.
    \item[(v)] For $0< \mu_1 < \mu_2$, If $\phi \in \{\phi_3,\phi_4, \phi_5, \phi_7, 
                \phi_8, \phi_9, \phi_{10} \}$, then $\phi(\mu_1,t) < \phi(\mu_2,t)$, and
                if $\phi \in \{\phi_1, \phi_2, \phi_6, \phi^{(p)}\}$, then  $\phi(\mu_1,t) > \phi(\mu_2,t)$.
    \item[(vi)] the following inequalities are true
            \begin{eqnarray*}
                    0 &\leq & \max \{0,x\} -\phi_1(\mu,x)  \leq \frac{\mu}{2},\\
                    0 &\leq & \max \{0,x\} -\phi_2(\mu,x)  \leq \mu,\\
                    0 &\leq & \phi_3(\mu,x)-\max \{0,x\}  \leq \mu \ln 2,\\
                    0 &\leq & \phi_4(\mu,x)-\max \{0,x\}  \leq \frac{\mu }{8},\\
                    0 &\leq & \phi_5(\mu,x)-\max \{0,x\}  \leq \mu, \\
                    0 &\leq & \max \{0,x\} -\phi_6(\mu,x)  \leq  \frac{\mu}{2} \ln 2,\\
                    0 &\leq & \phi_7(\mu,x)-\max \{0,x\}  \leq  \mu \ln 2, \\
                    0 &\leq & \phi_8(\mu,x)-\max \{0,x\}  \leq  \frac{\mu}{8}, \\
                    0 &\leq & \phi_9(\mu,x)-\max \{0,x\}  \leq  \frac{3\mu}{16}, \\
                    0 &\leq & \phi_{10}(\mu,x)-\max \{0,x\}  \leq  \frac{\mu}{\sqrt{2\pi}}, \\
                    0 &\leq & \max \{0,x\} -\phi^{(p)}(\mu,x)  \leq \frac{p-1}{p}\mu^{\frac{1}{p-1}}.
            \end{eqnarray*}
    \end{description}
 \end{proposition}

 \beginproof
 The proof is straightforward and we omit it.
 \endproof


From the Proposition \ref{approximationBound} and Figure \ref{All-Smooth}, we see that the local behavior of all 11 smoothing functions can be described as follow.
 \begin{equation}\label{relation-phi}
 \phi_2 \leq \phi_1  \leq \phi_6 \leq \phi^{(p)} \leq (x)_+  \leq \phi_4 \approx \phi_8  \leq \phi_9 \leq \phi_{10} \leq \phi_7 \approx
 \phi_3 \leq \phi_5.
 \end{equation}


 We can divide the above functions into three categories. The first class is approximated
 by the Plus function from above, and in this class we have the smoothing functions
 $\phi_4 , \phi_8  , \phi_9 , \phi_{10} , \phi_7 , \phi_3 , \phi_5$. The second type
 approximates the plus function from below and has the same value as the plus function when $x$ is less than $0$.
 This class contains smoothing function $ \phi_2 , \phi_1 ,  \phi^{(p)}$. The third class include
 only function $\phi_6$, which is characterized by an approximation from below but coincidences with only one zero point of plus function.


 \medskip


 Next, we recall some materials about first order differential equations (ODE):
 \begin{equation} \label{ODE}
 \dot{w}(t)=H(w(t)),\ w(t_0)=w_0\in \Re^n,
 \end{equation}
 where $H: \Re^n \to \Re^n$ is a mapping. We also introduce three kinds
 of stability that will be discussed later. These materials can be
 found in usual ODE textbooks.

 \medskip

%----------------------------------------------------------------------------------- Definition 2.1
 \begin{definition}
 A point $w^*=w(t^*)$ is called an equilibrium point or a steady
 state of the dynamic system (\ref{ODE}) if $H(w^*)=0.$ If there is
 a neighborhood $\Omega^*\subseteq \Re^n$ of $w^*$ such that
 $H(w^*)=0$ and $H(w)\neq 0$ $\forall w\in \Omega^*\setminus\{w^*\}$, then
 $w^*$ is called an isolated equilibrium point.
 \end{definition}

 \medskip

%---------------------------------------------------------------------------------------- Lemma 2.1
 \begin{lemma} \label{wdomaim}
 Assume that $H: \Re^n \to \Re^n$ is a continuous mapping. Then, for any
 $t_0>0$ and $w_0\in \Re^n$, there exists a local solution $w(t)$ for
 (\ref{ODE}) with $t\in [t_0,\tau)$ for some $\tau>t_0.$ If, in
 addition, $H$ is locally Lipschitz continuous at $x_0$, then the
 solution is unique; if $H$ is Lipschitz continuous in $\Re^n$, then
 $\tau$ can be extended to $\infty$.
 \end{lemma}


 If a local solution defined on $[t_0,\tau)$ cannot be extended to a
 local solution on a larger interval $[t_0,\tau_1)$, $\tau_1>\tau,$
 then it is called a maximal solution, and the interval $[t_0,\tau)$
 is the maximal interval of existence. Clearly, any local solution
 has an extension to a maximal one. We denote $[t_0, \tau(w_0))$ by
 the maximal interval of existence associated with $w_0$.


 \medskip

%---------------------------------------------------------------------------------------- Lemma 2.2
 \begin{lemma}\label{limmaxsolution}
 Assume that $H: \Re^n \to \Re^n$ is continuous. If $w(t)$ with
 $t\in [t_0, \tau(w_0))$ is a maximal solution and $\tau(w_0)<\infty$,
 then $\ds \lim_{t\uparrow\tau(w_0)}\|w(t)\|=\infty.$
 \end{lemma}

 \medskip

%----------------------------------------------------------------------------------- Definition 2.2
 \begin{definition} (Stability in the sense of Lyapunov)
 Let $w(t)$ be a solution for (\ref{ODE}). An isolated equilibrium point
 $w^*$ is Lyapunov stable if for any $w_0=w(t_0)$ and any  $\varepsilon >0$,
 there exists a $\delta>0$ such that $\|w(t)-w^*\|<\varepsilon$ for all
 $t\geq t_0$ and $\|w(t_0)-w^*\|<\delta$.
 \end{definition}

 \medskip

%----------------------------------------------------------------------------------- Definition 2.3
 \begin{definition} (Asymptotic Stability)
 An isolated equilibrium point $w^*$ is said to be asymptotic stable
 if in addition to being Lyapunov stable, it has the property that
 $w(t)\to w^*$ as $t\to \infty$ for all $\|w(t_0)-w^*\| < \delta$.
 \end{definition}

 \medskip

%------------------------------------------------------------------------------- Definition 2.4
 \begin{definition} (Lyapunov function)
 Let $\Omega \subseteq \Re^n$ be an open neighborhood of $\bar{w}$.
 A continuously differentiable function $V:\Re^n\to \Re$ is said to
 be a Lyapunov function at the state $\bar{w}$ over the set $\Omega$
 for equation (\ref{ODE}) if
 \begin{equation} \label{Lfunction}
 \left\{
 \begin{array}{ll}
 V(\bar{w})=0,\ V(w)>0,\ \forall w\in\Omega\setminus\{\bar{w}\}, \\
 \dot{V}(w)\leq 0,\ \forall w\in\Omega\setminus\{\bar{w}\}.
 \end{array}\right.
 \end{equation}
 \end{definition}

 \medskip

%----------------------------------------------------------------------------------------- Lemma 2.3
 \begin{lemma} \label{stablecondition}
 \begin{description}
 \item[(a)] An isolated equilibrium point $w^*$ is Lyapunov stable
            if there exists a Lyapunov function over some
            neighborhood $\Omega^*$ of $w^*$.
 \item[(b)] An isolated equilibrium point $w^*$ is asymptotically stable
            if there exists a Lyapunov function over some
            neighborhood $\Omega^*$ of $w^*$ such that $\dot{V}(w)<0,\
            \forall w\in\Omega^*\setminus \{w^*\}$.
 \end{description}
 \end{lemma}

 \medskip

%----------------------------------------------------------------------------------- Definition 2.4
 \begin{definition} (Exponential Stability)
 An isolated equilibrium point $w^*$ is exponentially stable if there
 exists a $\delta>0$ such that arbitrary point $w(t)$ of (\ref{ODE})
 with the initial condition $w(t_0)=w_0$ and $\|w(t_0)-w^*\|<\delta$
 is well defined on $[0,+\infty)$ and satisfies
 \[
 \|w(t)-w^*\|\leq ce^{-\omega t}\|w(t_0)-w^*\|,\quad \forall t\geq
 t_0,
 \]
 where $c>0$ and $\omega>0$ are constants independent of the initial
 point.
 \end{definition}

 \medskip

%------------------------------------------------------------------------------------------------------- Section 3
 \section{Smoothing neural network model for system of inequalities}

This section utilizes the smoothing function constructed in the 
previous section to construct a smooth neural network model for 
solving the  system of inequalities \eqref{INEQUALITY}.

Let,
\begin{equation}\label{Psi}
\Psi(x) = \frac{1}{2}\| [ f(x) ]_+ \|^2
\end{equation}
Then the system of inequalities \eqref{INEQUALITY}
can be transformed into an  unconstrained optimization
problem.
\begin{equation}\label{minPsi}
\min_{x\in \rn} \Psi(x).
\end{equation}
In order to overcome the nonsmoothness of the problem \eqref{minPsi},
smoothing technique is adopted.

For $\phi \in \{\phi_i,\phi^{(p)}\}$, $i=1,\cdots,10$, defined 
by \eqref{phi1} -- \eqref{phi10},
 we defined function $H:R^{n+1} \rightarrow R^{n+1}$ by

\[
H(\mu,x)=\left[
    \begin{array}{c}
        \mu, \\
        \Phi(\mu,f(x))
    \end{array}
\right]=
    \left[
    \begin{array}{c}
        \mu, \\
        \phi(\mu,f_1(x))\\
        \vdots \\
        \phi(\mu,f_n(x))
    \end{array}
    \right]
\]


Let,
\begin{equation}\label{Psimu}
\Psi_\mu(\mu,x) = \mu^2+\sum_{k=1}^{m}(\phi\left(\mu,f_k(x))\right)^2
=\frac{1}{2}\| H(\mu, x ) \|^2 ,
\end{equation}
Then, according to the above function $\Psi_\mu(\mu,x)$, the following
 unconstrained optimization problem is defined
\begin{equation}\label{minPsimu}
\min_{x\in \rn} \Psi_\mu(\mu,x).
\end{equation}

In other hand, we can think of $\Psi_\mu(\mu,x)$ and unconstrained 
problem \eqref{minPsimu} as being transformed from the following system of inequalities
\begin{equation*}
\left\{
\begin{array}{c}
f(x) \leq 0, \\
-\mu \leq 0.
\end{array}
\right.
\end{equation*}

The next theorem gives the relation beteween problem \eqref{minPsi} and
problem \eqref{minPsimu}.
We need following Lemma. In the following analysis, only smooth 
function $\phi = \phi_1$ is used, which is similar when other 
functions  $\phi_i$, $i=1,\cdots,10$ and $\phi^{(p)}$ are used.

\begin{lemma}\label{lemma-bound}
For all $x \in \rn$, $\mu \geq 0$, we have
\begin{equation}
\mu^2 \leq \Psi(x) - \Psi_\mu(\mu,x) \leq \mu^2 +\frac{n\mu}{2}.
\end{equation}
\end{lemma}


\begin{proof}
Just give the proof in the case of smoothing function $\phi = \phi_1$, 
the cases for other ten smoothing functions are similar.

From Proposition \ref{approximationBound}, we have
$$
0 \leq \max\{0,x\} - \phi(\mu,x) \leq \frac{\mu}{2},
$$
Then,
$$
0 \leq [f(x)]_+  - \phi(\mu,f(x)) \leq \frac{\mu}{2}.
$$
Sum of both sides and add $\mu^2$
$$
\mu^2 \leq \Psi(f(x))  - \Psi_\mu(\mu,f(x) \leq \mu^2 +\frac{n\mu}{2}.
$$
\end{proof}

\begin{theorem}\label{theorem31}
Let the sequence $\mu_k \rightarrow 0$, $\mu_k \geq 0$, and $x_k$ is an 
optimal solution of problem \eqref{minPsimu}, Assume that $x^*$ be an
accumulating point of the sequence ${x_k}$. Then
$x^*$ be an optimal solution to problem \eqref{minPsi},
\end{theorem}

\begin{proof}
From Lemma \ref{lemma-bound}, for every $k$, we have
$$
\mu_k^2 \leq \Psi(f(x))  - \Psi_\mu(\mu_k,f(x) \leq \mu_k^2 +\frac{n\mu_k}{2}.
$$
Then,
$$
\min_{x\in \rn} \{ \mu_k^2 \} \leq  \min_{x\in \rn} \{ \Psi(f(x))  - x_k \} \leq \min_{x\in \rn} \{ \mu_k^2 +\frac{n\mu_k}{2} \}.
$$
Let's take $k \rightarrow \infty$, we have
$$
\lim_{k \rightarrow \infty}\Psi(f(x))=  \lim_{k \rightarrow \infty}x_k = x^*.
$$
\end{proof}

\begin{theorem} \label{theorem32}
Let $x^*$ and $\bar{x}$ are the optimal solutions of problem \eqref{minPsi}
and \eqref{minPsimu}, respectively.
Then
$$
0 \leq \Psi(f(x^*))  - \Psi_\mu(\mu,f(\bar{x})) \leq \mu\frac{n+2\mu}{2}.
$$
\end{theorem}

\begin{proof}
From Lemma \ref{lemma-bound} and $x^*$ and $\bar{x}$ are the optimal 
solutions of problem \eqref{minPsi} and \eqref{minPsimu}, we have
\[
\mu_k^2 \leq \Psi(f(x^*))  - \Psi_\mu(\mu_k,f(x^*)),
\]
\[
\Psi_\mu(\mu_k,f(\bar{x})  \leq \Psi_\mu(\mu_k,f(x^*)).
\]
Then,
\[
\mu_k^2 + \Psi_\mu(\mu_k,f(\bar{x}))  \leq \Psi(f(x^*))
\]
That is
\begin{equation}\label{theoremInqua}
\Psi_\mu(\mu_k,f(\bar{x}))  + \Psi(f(x^*)) \leq 0.
\end{equation}
From Lemma \ref{lemma-bound}, $x^*$ and $\bar{x}$ be an optimal 
solutions of problem \eqref{minPsi}, we have
\[
\Psi(x^*) \leq \Psi(\bar{x}) \leq \Psi(\mu,\bar{x}) + \mu\frac{n+2\mu}{2},
\]
Therefor, we obtain
\begin{equation}\label{theoremIqua2}
 \Psi(x^*) - \Psi(\mu, x^*)  \leq  \mu\frac{n+2\mu}{2}
\end{equation}

Since \eqref{theoremInqua},\eqref{theoremIqua2}, there have
\[
 0 \leq \Psi(x^*) - \Psi(\mu, x^*)  \leq  \mu\frac{n+2\mu}{2}.
\]
\end{proof}

The Theorem \ref{theorem31} and Theorem \ref{theorem32} show that 
the solution to smooth problem \eqref{minPsimu} is also optimal
solution to the nonsmooth problem \eqref{minPsi} in small error.

In order to describe the relationship between the optimal  solution 
to  \eqref{minPsimu} and a feasible solution of the original system of
inequalities \eqref{INEQUALITY}, we need to add extra conditions to 
the function $f(s)$ of problem \eqref{INEQUALITY}. 
When the function $f(x)$ is convex, we can characterize the relation 
between the solution of the original system of  inequalities \eqref{INEQUALITY} 
and the solution of the smooth problem \eqref{minPsimu}. Import the
Theorem \ref{distance} through the following lemma for error bound \cite{M1995,M1985}.

\begin{lemma}\label{errorbound}
Suppose that the solution set $X$ of  \eqref{INEQUALITY} is nonempty, 
$f(x)$ be a differentiable convex function on  $\rn$ and satisfy the 
Slater and asymptotic constraint qualification \cite{M1995}.
Then there exist $\tilde{x} \in X$, such that
    $$ \| x- \tilde{x}\| \leq C\|(f(x))_+\|,$$
where $C$ is a constant dependent on $f(x)$ \cite{M1995}.
\end{lemma}

\begin{theorem}\label{distance}
  Suppose that the conditions of Lemma \ref{errorbound} are satisfied 
and $\bar{x}(\mu)$ be solution to \eqref{minPsimu}.
\begin{enumerate}
    \item[(a)] There exist $\tilde{x}(\mu) \in X$, such that
                $$ \| \bar{x}(\mu) - \tilde{x}(\mu)\| \leq rC\sqrt{n}.$$
                where $C$ is a constant and same as Lemma \ref{errorbound} and
                \begin{equation*}
                 r=\left\{
                    \begin{array}{cc}
                        \frac{p-1}{p}\mu^{\frac{1}{p-1}}, & \phi= \phi^{(p)}\\
                        \mu, & \text{other},\\
                     \end{array}\right.
                \end{equation*}

    \item[(b)] If the  Slater constraint qualification  is satisfied 
                by $f({x}) \leq 0$, then there exist a $\bar\mu >0$ such that for 
                any $\mu \leq \bar\mu$, $x(\mu)$ solve the convex 
                inequalities \eqref{INEQUALITY} exactly.
\end{enumerate}
\end{theorem}

\begin{proof}
Consider that $\bar{x}$ is optimal solution of \eqref{minPsimu}
and (i) of Proposition \eqref{approximationBound},
 \begin{equation}\label{maxPsimu}
    2\Psi_\mu(\bar{x}) \leq 2\Psi_\mu(\tilde{x}) = \| \Phi(\mu, f(\tilde{x}))\|^2 
    +\mu^2 \leq  \|\Phi(\mu, 0)\|^2 + \mu^2.
\end{equation}
 We divide the smoothing function $\phi(\mu,t)$ into three classes and 
discuss them separately.
\begin{description}
    \item[Case (i)] If $\phi \in \{\phi_3, \phi_4, \phi_5, \phi_7, 
                    \phi_8, \phi_9, \phi_{10}\}$, then $\phi(\mu,t) \geq (t)_+$, for all $\mu \geq 0$.
                    From Lemma \ref{errorbound}, 
                \begin{eqnarray*}
                     \| \bar{x}(\mu) - \tilde{x}(\mu)\| &\leq& C\|(g( \bar{x}))_+\| \leq  
                      C\|\Phi(\mu,f( \bar{x}))\| = C\sqrt{2\Psi_\mu(\bar{x})-\mu^2}\\
                         &\leq&   C\|\Phi(\mu, 0)\| =C\sqrt{n}\phi(\mu,0) \leq \mu C\sqrt{n}.
                \end{eqnarray*}

    \item[Case (ii)] If $\phi \in \{\phi_1, \phi_2,\phi_6 \}$, then $\phi(\mu,t) 
                        \leq (t)_+$, for all $\mu \geq 0$.
                 Let
                 \begin{equation*}
                     q(\mu)=\left\{
                    \begin{array}{cc}
                            \mu/2, & \phi= \phi_1, \\
                            \mu, & \phi= \phi_2, \\
                             % (\mu\log2)/2, & \phi= \phi_6, \\
                            %  \frac{p-1}{p}\mu^{\frac{1}{p-1}}, &\phi= \phi^{(p)}.
                    \end{array}
                    \right.
                 \end{equation*}
                Then,  $\phi(\mu,t) +  2q(\mu) \geq (t)_+$, for all $\mu \geq 0$.
                 From Lemma \ref{errorbound}, we have
                \begin{eqnarray*}
                        \| \bar{x}(\mu) - \tilde{x}(\mu)\| 
                        &\leq&  C\|(f( \bar{x}))_+\| \leq  C( \|\Phi(\mu,f( \bar{x})) +2q(\mu)\|)  \\
                        &\leq&  C\|\Phi(\mu,f( \bar{x}))\| + 2C\sqrt{n}q(\mu) 
                                \leq C\|\Phi(\mu,0)\| + 2C\sqrt{n}q(\mu)  \\
                        &=& C\sqrt{n}\phi(\mu,0) + 2C\sqrt{n}q(\mu) = 2C\sqrt{n}q(\mu))\\
                        &\leq& 2\mu C\sqrt{n}.
                        \end{eqnarray*}
     \item[Case (iii)] If $\phi =\phi^{(p)}$, similar to the analysis in case (ii), we can draw the following result.
                \begin{eqnarray*}
                % \nonumber % Remove numbering (before each equation)
                    \| \bar{x}(\mu) - \tilde{x}(\mu)\|
                    &\leq&   2\sqrt{n}C\frac{p-1}{p}\mu^{\frac{1}{p-1}}\\
                    &\leq&  .
                \end{eqnarray*}
  \end{description}
\end{proof}

The Theorem \ref{distance} say that an approximate solution of a feasible 
solutions can be obtained by solving the smoothing model \eqref{minPsimu} in small error.

\medskip

Next, we construct a gradient neural network model to solve the unconstrained 
optimization problem \eqref{minPsimu}.
 Let $w=(\mu,x)$, it  is clear  that if
 $w^{*}\in \Re_{++}\times \Re^{n}$ solves $H(w)=0$, then $w^{*}$ solves
 $ \nabla \Psi(w) =0$ because
 \begin{eqnarray*}
 &  &  H(w^{*})=0
  \Longleftrightarrow   \Psi(w^{*})=0 \\
 & \Longleftrightarrow &  w^{*} \ \  \mbox{minimizes} \ \ \Psi_\mu(w^{*})
   \Longrightarrow         \nabla \Psi_\mu(w^{*}) = 0.
 \end{eqnarray*}

 \medskip

 Let $w(t)=(\mu(t),x(t))$, applying the gradient approach to the
 minimization of the energy function \eqref{Psimu},
 we obtain the system of differential equation:
 \begin{equation} \label{ANN}
    \left \{
    \begin{array}{l}
                \ds \frac{dw(t)}{dt} = -\rho \, \nabla  \Psi_\mu(w(t))
                = -\rho \nabla H_i(w(t))^{\mathrm{T}} H_i(w(t)), \\
                w(t_0) = w_0.
    \end{array}
    \right.
 \end{equation}
 where $w_0=(\mu_0, x_0)^{\mathrm{T}} \in \Re^{n+1}$, $\rho>0$
 is a time scaling factor.

 \medskip

%----------------------------------------------------------------------------------- Assumption 3.1
 \begin{assumption} \label{Assumption3.1}
 The function $f(x)$ continuously  differentiable in $D \in \rn$.
 \end{assumption}
 %----------------------------------------------------------------------------------- Assumption 3.2
 \begin{assumption} \label{Assumption3.2}
 $\forall \mu > 0$, the Jacobian matrix J of function  $f(x)$ is nonsingular in $D \in \rn$.
 \end{assumption}

In practice, if the function $f(x)$ is a monotone function,
i.e.,$(x-y)^{T} (f(x)-f(y)) \geq 0, x,y \in \rn$,
then $\nabla f(x)$ is a positive semi-definite matrix for
any $x \in \rn$ and, hence, Assumption \ref{Assumption3.2} is
satisfied.

 \medskip



%---------------------------------------------------------------------------------- Proposition 3.1
\begin{proposition} \label{Proposition 3.1}
    Let smoothing function $\phi(\mu,t) \in \{\phi_1(\mu,t),\cdots,
    \phi_{10}(\mu,t),\phi^{(p)}(\mu,t) \}$, and the assumption
    \ref{Assumption3.1},\ref{Assumption3.2} are satisfied. Then, we have
    \begin{description}
    \item[(i)] $H_i (\mu, x)=0 $ if and only if $x$ solves the system of inequalities (\ref{INEQUALITY});
     \item[(ii)] $H$ is continuously differentiable on
            $\Re ^{n+1} \backslash \left\{\mathbf{0}\right\}$
            with the Jacobian matrix given by
            \begin{eqnarray}\label{DHi}
            \nabla H (\mu, x):= \left[
            \begin{array}{cc}
            1 & \mathbf{0} \\
            A & B^\mathrm{T}J
            \end{array}
            \right],
            \end{eqnarray}
            where,   $J$ is the Jacobian matrix of $f(x)$ and
            \begin{eqnarray*}
            A &:=& \left( \frac{\partial \phi(\mu, f_1(x))}
            {\partial \mu},\cdots,  \frac{\partial \phi(\mu, f_n(x))}{\partial \mu}\right)^{\mathrm{T}}, \\
            B &:=& \left( \frac{\partial \phi(\mu, f_1(x))}
            {\partial t},\cdots,  \frac{\partial \phi(\mu, f_n(x))}{\partial t}\right)^{\mathrm{T}}.
            \end{eqnarray*}
    \end{description}
 \end{proposition}

 \beginproof
 The arguments are straightforward and we omit them.
 \endproof

 \medskip

%---------------------------------------------------------------------------------- Proposition 3.4
 \begin{proposition} \label{propsition 3.2}
 Let $\Psi:\rn \rightarrow \Re_+ $ be given by \eqref{minPsimu} and
 assumption \ref{Assumption3.1},\ref{Assumption3.2} are satisfied.
 Then, the following
 results hold.
 \begin{description}
 \item[(i)] $\Psi(\mu,x) \geq 0, \forall (\mu, t) \in \Re_{++} \times \Re $ and
            $\Psi(\mu,x)= 0$ if and only if $x$ solve the system of inequalities \eqref{INEQUALITY}.

 \item[(ii)] The function  $\Psi(\mu, x)$  is continuously differentiable on
            $\Re ^{n+1} \backslash \left\{\mathbf{0}\right\}$ with
            \[
            \nabla \Psi(\mu,x)=\nabla H^\mathrm{T}H,
            \]
            where $\nabla H $ is the Jacobian of $H(\mu,x)$.
 \item[(iii)] The function $\Psi(w(t))$ is nonincreasing with respect to $t$.
 \end{description}
 \end{proposition}
 \beginproof
 Parts (i)-(ii) follow from Proposition \ref{Proposition 3.1} immediately.

 \medskip
 \noindent
 For part (iii), we observe that
 \begin{eqnarray*}
 \frac{d \Psi(w(t))}{d t}
 &=& \left \lla \frac{dw}{dt},\nabla \Psi(\mu,u) \right \rla \\
 &=& \left \lla -\rho \, \nabla \Psi(\mu,u) ,
     \nabla \Psi(\mu,u) \right\rla \\
 &=& -\rho \, \left \| \nabla \Psi(\mu,u)  \right \|^2   < 0,
 \end{eqnarray*}
 for all $u \in \Omega \setminus \{u^{*} \}$. Then, the desired result follows.
 \endproof


 \medskip




%---------------------------------------------------------------------------------------------- Section 4
 \section{Stability and Existence}



 In this section, we first give the relation between the solution of system of
 inequalities \eqref{INEQUALITY} and the equilibrium point of neural network \eqref{ANN}.
 Then, we discuss the issues of the stability and the the solution trajectory of
 the neural network \eqref{ANN}.

 \medskip

%---------------------------------------------------------------------------------------- Lemma 4.2
 \begin{lemma}\label{lmaeq}
 Let $x^{*}$ be a equilibrium of the neural network (\ref{ANN}) and suppose that
 assumption \ref{Assumption3.2} are satisfied. Then $x^{*}$ solves the
 system (\ref{INEQUALITY}).
 \end{lemma}
 \beginproof
 Since $  \nabla \left( \Psi(w^{*}) \right)= \nabla H_i^\mathrm{T}H_i(w^{*})$
 and from the assumption \ref{Assumption3.2}, Proposition \ref{Proposition 2.2} and \ref{Proposition 3.1}
  obtain  $\nabla H$ is nonsingular. It is clear to see that
 $$\nabla \left( \Psi(w^{*}) \right)=0, $$ if and only if $H_i(w^{*})=0$.
 \endproof

 \medskip

%-------------------------------------------------------------------------------------- Theorem 4.1
 \begin{theorem}
 \begin{description}
 \item[(a)] For any initial point $w_0=w(t_0)$, there exists a unique
            continuously maximal solution $w(t)$ with $t\in [t_0, \tau)$
            for the neural network (\ref{ANN}).
 \item[(b)] If the level set
            ${\cal L}(w_0):= \left\{ w \, | \, \|H_i(w)\|^2 \leq \|H(w_0)\|^2 \right\}$
            is bounded, then $\tau$ can be extended to $\infty$.
 \end{description}
 \end{theorem}
 \beginproof
 This proof is exactly the same as the one in \cite[Proposition 3.4]{SCK12},
 so we omit it here.
 \endproof

 \medskip

 Now, we are going to analyze the stability of an isolated equilibrium $x^*$
 of the neural network (\ref{ANN}), which is to assert that $\nabla\Psi(x^{*} ) =0$
 and $\nabla\Psi(x)\neq0$ for $x\in\Omega\setminus\{x^{*}\}$,
 $\Omega$ is a neighborhood of $x^*$.

 \medskip

%-------------------------------------------------------------------------------------- Theorem 4.2
 \begin{theorem}\label{asym_stable}
 If assumption \ref{Assumption3.1},\ref{Assumption3.2} are satisfied, then the isolated
 equilibrium $x^*$ of the neural network (\ref{ANN}) is asymptotically stable,
 and hence Lyapunov stable.
 \end{theorem}
 \beginproof
 We consider the Lyapunov function $\Psi(w):\Omega \to \Re$ defined by \eqref{energy}.
 First, it is clear that $\Psi(x)\geq 0$ and from (a) of Proposition \ref{Proposition 3.1}
 we have $\Psi(\cdot)$ is continuously differentiable.
 Considering assumption \ref{Assumption3.2}, Proposition \ref{Proposition 2.2} and \ref{Proposition 3.1}
 we obtain $\nabla H(w^*)$ is nonsingular.
 Then applying $\nabla H(x^*)$ and Lemma \ref{lmaeq}, we have $H(w^*)=0$ and $\Psi(w^*)=0$.
 Furthermore, if $\Psi(w)=0$ on $\Omega$, then $H(w)=0$ and hence
 $\nabla \Psi=0$ on $\Omega$. This yields that $w=w^*$
 since $w^*$ is isolated.

 \medskip
 \noindent
 Secondly, consider the (b) of Proposition \ref{Proposition 3.1} and Lemma \ref{stablecondition},
 it says the isolated equilibrium $x^*$ is asymptotically stable, and hence
 is Lyapunov stable.
 \endproof

 \medskip

%-------------------------------------------------------------------------------------- Theorem 4.3
 \begin{theorem}
 If assumption \ref{Assumption3.1},\ref{Assumption3.2} are satisfied, then the isolated
 equilibrium $x^*$ of the neural network (\ref{ANN}) is exponentially stable.
 \end{theorem}
 \beginproof
 Let $\Omega = \Re_{++} \times \Re^n$, it is clear that $H(\cdot)$ is continuously
 differentiable, which implies
 \begin{equation} \label{Hu}
 H(w)=H(w^*)+\nabla H(w^*)^T(w-w^*) + o(\|w-w^*\|), \quad \forall \, w \in \Omega.
 \end{equation}
 Let $g(t):=\frac{1}{2}\|w(t)-w^*\|^2$ and we compute the derivative of $g(t)$
 as below:
 \begin{eqnarray*}
 & & \ds \frac{d g(t)}{d t} = \left( \frac{dw}{dt} \right)^T (w(t)-w^*)
     = -\frac{\rho}{2} \, \nabla(\|H(w)\|^2)^T (w(t)-w^*) \\
 &=& -\rho \, (\nabla H(w) \cdot H(w))^T (w(t)-w^*) = -\rho \, H(w)^T \nabla H(w)^T (w(t)-w^*) \\
 &=& -\rho \, (w(t)-w^*)^T \nabla H(w^*) \nabla H(w)^T (w(t)-w^*)
     -\rho \, o(\|w-w^*\|)^T \nabla H(w)^T (w(t)-w^*),
 \end{eqnarray*}
 where the last equality is due to (\ref{Hu}). To proceed, we claim two assertions.
 First, we claim that
 \(
 (w-w^*)^T \nabla H(w^*) \nabla H(w)^T(w-w^*)\geq \kappa ||w-w^*||^2,
 \)
 for some $\kappa$. To see this,  from the Assumption \ref{Assumption3.1},\ref{Assumption3.2}, Proposition \ref{Proposition 2.2}
 and Proposition \ref{Proposition 3.1}, we know $\nabla H(w)$ is nonsingular and
 $H$ is a continuously differentiable function, which implies the matrix
 $\nabla H(w^*)\nabla H(w^*)^T$ is symmetric and positive semi-definite. Hence,
 we have
 $(w-w^*)^T \nabla H(w^*) \nabla H(w^*)^T(w-w^*) \geq \kappa_1 \|w-w^*\|^2>0$
 over $\Omega \setminus \{w^{*}\}$ for some $\kappa_1\geq0$. Then, by the continuity
 of $\nabla H(\cdot)$, we can conclude that
 \[
 (w-w^*)^T \nabla H(w^*) \nabla H(w)^T(w-w^*) \geq \kappa \|w-w^*\|^2 >0,
 \quad \mbox{for some $\kappa \geq0$}.
 \]

 \medskip
 \noindent
 Secondly, we claim that
 \[
 - \rho \, o(\|w-w^*\|)^T \nabla H(w)^T(w(t)-w^*)\leq
 \varepsilon \|w-w^*\|^2, \quad \mbox{for some $\varepsilon >0$}.
 \]
 This is because that
 \[
 \frac{\big|- \rho \, o(\|w-w^*\|)^T \nabla H(w)^T(w(t)-w^*)\big|}{\|w-w^*\|^2}
 \leq \rho \|\nabla H(w)\| \left( \frac{\|o(\|w-w^*\|)\|}{\|w-w^*\|} \right),
 \]
 where the right-hand side vanishes eventually. Thus, it yields that
 \[
 - \rho \, o(\|w-w^*\|)^T \nabla H(w)^T(w(t)-w^*)\leq
 \varepsilon \|w-w^*\|^2, \quad \mbox{for some $\varepsilon >0$}.
 \]
 Now, from the above two assertions and noting that $g(t)=\frac{1}{2}\|w(t)-w^*\|^2$,
 we have
 \[
 \frac{d g(t)}{d t} \leq 2 (-\rho\kappa+\varepsilon)g(t),
 \]
 which gives
 \[
 g(t)\leq e^{2(-\rho\kappa+\varepsilon)t} g(t_0).
 \]
 Thus, we have
 \[
 \| w(t)-w^*\| \leq e^{(-\rho\kappa+\varepsilon)t} \, \|w(t_0)-w^*\|,
 \]
 which says $w^*$ is exponentially stable as we can set $\rho$ larger enough
 such that $-\rho\kappa+\varepsilon<0$. Then, the proof is complete.
 \endproof

 \medskip




%---------------------------------------------------------------------------------------------- Section 5
 \section{Numerical Results}

 In order to demonstrate the effectiveness of the proposed neural network, we
 test several examples for our neural network \eqref{ANN} in this section. The
 numerical implementation is coded by Mathematica 12.0 and the ordinary
 differential equation solver adopted is NDSolve[ ], which uses an Runge-Kutta (2,3)
 formula. The stopping criteria for all test problems is $\frac{1}{2}\|H(u) \|^2 < 10^{-6}$.
 The following test problems are employed from \cite{Chen2016,HZ2008}.

%------------------------------------------------------------------------------------------------------------------------- Example 1
 \begin{example}\label{ex1}
 Consider the following system of inequalities
 \[
f_1(x)=x_1^2 + x_2^2 -1 +  \varepsilon,
\ f_2(x)= -x_1^2 - x_2^2 +(0.999)^2 + \varepsilon
 \]
 \end{example}


\begin{figure}[ht]
\centering
\begin{tikzpicture}
\scope[nodes={inner sep=0,outer sep=0}]
\node[anchor=south east] (a)
  {\includegraphics[width=7cm]{exmaple-1-Trajac-Phi5.pdf}};
\node[anchor=south west] (b)
  {\includegraphics[width=7cm]{exmaple-1-Hnorm-Phi5.pdf}};
\node[anchor=north east] (c)
  {\includegraphics[width=7cm]{exm01-bar1.pdf}};
\node[anchor=north west] (d)
  {\includegraphics[width=7cm]{exm01-bar2.pdf}};
\endscope
\foreach \n in {a,b,c,d} {
  \node[anchor=north west,fill=yellow!50] at (\n.north west) {(\n)};
}
%\node[anchor=north east,fill=green] at (e.south east) {(e)};
%\node[anchor=south west,fill=green] at (f.south west) {(f)};
\end{tikzpicture}
\caption{Computing results of Example \ref{ex1}}\label{FigEx01}
\end{figure}



 For the Example \ref{ex1}, the parameter list of neural network \ref{ANN}
 are following $\rho = 1, c=1, dt=0.2, x_0 = (0,5), s_0 = f(x_0),
 \varepsilon = 10^{-7} $ . Example \ref{ex1} can be solved successfully by
  neural network \eqref{ANN} with smoothing function $\phi_1, \cdots , \phi_8$,
  but not by the function  $\phi_9,  \phi_{10}, \phi^{(p)}(p=1.3, 1.6.2.5)$.
 For the successfully solved functions $\phi_1, \cdots , \phi_8$, there is
 no difference for the trajectories and the decline curves of the energy functions.
 Therefor, the figure (a)and (b) of Figure \ref{FigEx01} depict the trajectory behaviour and
 convergence behaviour of neural network \eqref{ANN} with only one smoothing
 function $\phi_5$. However, The (c) and (d) of Figure  \ref{FigEx01} show that in
 terms of computing time, smoothing function $\phi_1, \phi_4, \phi_5, \phi_8$
 very efficiency and followed by $\phi_2$. The $\phi_3, \phi_7$ function
 performs worst in computation time.



 %------------------------------------------------------------------------------------------------------------------------- Example 2
 \begin{example}\label{ex2}
 Consider the following system of inequalities
 \[
 f_1(x) =\sin(x_1)+ \varepsilon, f_2(x) = -\cos(x_2)+ \varepsilon,
 f_3(x) = x_1^2 + x_2^2 + x_3^2 + \varepsilon
 \]
 \end{example}


 The parameter list are following $\rho = 1, c=1, dt=0.2, x_0 = (0,0,0),
  s_0 = f(x_0), \varepsilon = 10^{-7} $ .
 Except for smoothing function $\phi_9$, all the other 12 smoothing functions
 can solve example \ref{ex2} successfully. The higher power of $\mu$ is used
 as the denominator and the small $\mu$ causes the function $\phi_9$ to be
 unable to solve successfully. The figure (a), (b) of Figure \ref{FigEx02} depict the
 trajectory behaviour and convergence behaviour of neural network \eqref{ANN}
 in example \ref{ex2}.

 These functions differ little in terms of performance, such as the iteration
 number, the value of $\frac{1}{2}\|H(u(t))\|^2$ when  algorithm terminates.
 But for computing time there are obviously difference. It can be seen from
 the time bar chart (a) of Figure \ref{FigEx02} that the computing time
 efficiency of function $\phi_3,\phi_4,\phi_7$ are much worse than other
 functions. The bar chart (b) of Figure  \ref{FigEx02} show the
 difference between other smoothing functions. The best one is $\phi_5$
 and following $\phi^{(p)},(p=1.3,1.6.2.5)$, $\phi_1$, $\phi_6$, $\phi_8$.

\begin{figure}[ht]
\centering
\begin{tikzpicture}
\scope[nodes={inner sep=0,outer sep=0}]
\node[anchor=south east] (a)
  {\includegraphics[width=7cm]{exmaple-2-Trajac-Phi5.pdf}};
\node[anchor=south west] (b)
  {\includegraphics[width=7cm]{exmaple-2-Hnorm-Phi5.pdf}};
\node[anchor=north east] (c)
  {\includegraphics[width=7cm]{exm02-bar1.pdf}};
\node[anchor=north west] (d)
  {\includegraphics[width=7cm]{exm02-bar2.pdf}};
\endscope
\foreach \n in {a,b,c,d} {
  \node[anchor=north west,fill=yellow!50] at (\n.north west) {(\n)};
}
%\node[anchor=north east,fill=green] at (e.south east) {(e)};
%\node[anchor=south west,fill=green] at (f.south west) {(f)};
\end{tikzpicture}
\caption{Computing results of Example \ref{ex2}}\label{FigEx02}
\end{figure}



 %-------------------------------------------------------------------------------------- Example 5.1
 \begin{example}\label{ex3}
 Consider the following system of inequalities
 \begin{eqnarray*}
 f_1(x) &=&\sin(x_1)+ \varepsilon, f_2(x) = -\cos(x_2)+ \varepsilon,
 f_3(x) = x_1 - 3\pi+x_3^2 + \varepsilon,\\
 f_4(x) &=&x_2 - \frac{\pi}{2} -2+x_4^2 + \varepsilon,
 f_5(x) = -x_1 - \pi + x_5^2 + \varepsilon,
 f_6(x) = -x_2 - \frac{\pi}{2} + x_6^2 + \varepsilon,
 \end{eqnarray*}
 \end{example}



 The parameter list are following $\rho = 1, c=100, dt=1, x_0 = (0,0,0,0,0,0),
  s_0 = f(x_0),\varepsilon = 10^{-7} $ .
 Except for smoothing function $\phi_9,\phi^{(p)},(p=1.3,1.6,2.5)$, all the
 other 9 smoothing functions can solve  example \ref{ex3} successfully. For
 this example the iteration number listed in Table \ref{tableEx03}. The figure
 (a) and (b) of Figure  \ref{FigEx03} depict the trajectory behaviour and convergence behaviour
 of neural network \eqref{ANN} with $\phi_5$ in example \ref{ex3}. The trajectory
 behaviour and convergence behaviour  of neural network \eqref{ANN} with other
 smoothing functions are very similar to Figure \ref{FigEx03},  so omit
 these figures.

 \begin{table}[ht]
  \centering
  \begin{tabular}{c c c c c c c c c c}
     \hline
     % after \\: \hline or \cline{col1-col2} \cline{col3-col4} ...
     $\phi$ & $\phi_1$ & $\phi_2$ & $\phi_3$ & $\phi_4$ & $\phi_5$ &
     $\phi_6$ & $\phi_7$ & $\phi_8$ &$\phi_{10}$ \\
      \hline
     N & 45 & 45 & 45 & 82 & 82 & 82 & 82 & 82 & 112 \\
     \hline
   \end{tabular}
  \caption{Iteration number with different $\phi_i$ in
  Example \ref{ex3}}\label{tableEx03}
\end{table}

 For the computing time there are obviously difference between 9 smoothing function.
 It can be seen from the time bar chart (a) of Figure \ref{FigEx03} that the
 computing time efficiency  of function $\phi_3,\phi_4,\phi_7$ are much worse than
 other functions. The  bar chat(b)  of  Figure  \ref{FigEx03} show the
 difference between other smoothing functions. The best  one is $\phi_5$ and
 following $\phi^{(p)},(p=1.3,1.6.2.5)$ , $\phi_1$, $\phi_6$,$\phi_8$.


\begin{figure}[ht]
\centering
\begin{tikzpicture}
\scope[nodes={inner sep=0,outer sep=0}]
\node[anchor=south east] (a)
  {\includegraphics[width=8cm]{exmaple-3-Trajac-Phi5.pdf}};
\node[anchor=south west] (b)
  {\includegraphics[width=8cm]{exmaple-3-Hnorm-Phi5.pdf}};
\node[anchor=north east] (c)
  {\includegraphics[width=8cm]{exm03-bar1.pdf}};
\node[anchor=north west] (d)
  {\includegraphics[width=8cm]{exm03-bar2.pdf}};
\endscope
\foreach \n in {a,b,c,d} {
  \node[anchor=north west,fill=yellow!50] at (\n.north west) {(\n)};
}
%\node[anchor=north east,fill=green] at (e.south east) {(e)};
%\node[anchor=south west,fill=green] at (f.south west) {(f)};
\end{tikzpicture}
\caption{Computing results of Example \ref{ex3}}\label{FigEx03}
\end{figure}

%---------------------------------------------------------------------------------------------- Section 6
 \section{Concluding Remarks}


 The main contents of this paper are as follows. Firstly, 11 kinds of smooth functions
 of projection functions are generated by using different mathematical methods.
 Some of these functions appear in the literature, but functions such as the $\phi^{(p)}$
 function have not been found in the previous literature.
 Secondly, a gradient neural network is constructed based on these smoothing functions and used
 for solving inequalities for the first time. Finally, Numerical experiments not only show that
 neural network  (\ref{ANN}) can effectively solve system of inequalities \eqref{INEQUALITY},
 but also summarize which smoothing functions are more effective in the neural network (\ref{ANN}).
 In general, we can conclude that the smoothing functions $\phi_3$,
 $\phi_7$, $\phi_9$ , $\phi_{10}$ performs poorly in neural networks (\ref{ANN}).
 The common feature of these functions is that their expressions are complex and
 contain higher degree term of $\mu$ as the denominator like $\phi_9$.
 On the other hand, function $\phi_1$, $\phi_2$, $\phi_5$, $\phi_5$ are computationally efficient
 in the neural network (\ref{ANN}).
 This means that we should make  the structure as simple as possible and not contain
 terms that vary greatly or very little with $\mu \rightarrow 0$.
 This is a very interesting discovery which may be helpful in other contexts. One of future
 directions is to check whether such phenomenon occurs in other types of algorithms.



 \medskip





%--------------------------------------------------------------------------------------------- References
 \begin{thebibliography}{1}
  \bibitem{fy18}
 {\sc Fan X, Yan Q},
 {\em Solving system of inequalities via a smoothing homotopy method},
  Numerical Algorithms, 2018: 1-10.

 \bibitem{BP93}
 {\sc A.\ Bouzerdoum and T.R.\ Pattison},
 {\em Neural network for quadratic optimization with bound constraints},
 IEEE Trans.Neural Networks, vol. 4, pp. 293--304, 1993.

 \bibitem{BM12}
 {\sc A.\ Beck and M.\ Teboulle},
 {\em Smoothing and first order methods: a unified framework},
 SIAM Journal on Optimization, vol. 22, pp. 557--580,2012.

 \bibitem{BC2016}
 {\sc H. Bauschke and P. Combettes},
 {\em  Convex Analysis and Monotone Operator Theory in Hilbert Spaces},
 New York, Springer, 2016.

% \bibitem{CQZ11}
% {\sc L.\ Caccetta, B.\ Qu, and G.-L.\ Zhou},
% {\em A globally and quadratically convergent method for absolute value equations},
% Computational Optimization and Applications, vol. 48, pp. 45--58, 2011.

% \bibitem{CKLW16}
% {\sc J.-S.\ Chen, C.-H.\ Ko, Y.-D.\ Liu, and S.-P.\ Wang},
% {\em New smoothing functions for solving a system of equalities and inequalities},
% Pacific Journal of Optimization, vol. 12, pp. 185--206, 2016.

 \bibitem{CM96}
 {\sc C.\ Chen and O.L.\ Mangasarian}
 {\em A class of smoothing functions for nonlinear and mixed complementarity problems},
 Computational Optimization and Applications, vol. 5, pp. 97--138, 1996.

  \bibitem{Chen2016}
 {\sc J-S Chen, C-H Ko, Y-D Liu, and S-P Wang},
 {\em New smoothing functions for solving a system of equalities and inequalities},
  Pacific Journal of Optimization, vol. 12, no. 1, pp. 185-206, 2016.

  \bibitem{CH1993}
  {\sc Bintong Chen, P. T. Harker},
  {\em  A non-interior-point continuation method for linear com- plementarity problems},
   SIAM Journal on Matrix Analysis and Applicatons, 14:1168-1190, 1993.

 \bibitem{CU93}
 {\sc A.\ Cichochi and R.\ Unbenhauen}
 {\em Neural Network for Optimization and Signal Processing},
 John Wiley \& Sons, Inc., 1993.

 \bibitem{HT85}
 {\sc J.J. Hopfield and D.W.\ Tank},
 {\em Neural computation of decision in optimization problems},
 Biological Cybernetics, vol. 52 pp. 141--152, 1985.

\bibitem{HZ2008}
 {\sc Huang, Z.H., Zhang, Y., Wu, W.},
 {\em A smoothing-type algorithm for solving system of inequalities},
  J. Comput. Appl. Math. 220, 355-363, 2008.

% \bibitem{HH10}
% {\sc S.-L.\ Hu and Z-H.\ Huang},
% {\em A note on absolute value equations},
% Optimization Letters, vol. 4, pp. 417--424, 2010.

% \bibitem{HHZ11}
% {\sc S.-L.\ Hu, Z.-H.\ Huang, and Q.\ Zhang},
% {\em A generalized Newton method for absolute value equations associated with second order cones},
% Journal of Computational and Applied Mathematics, vol. 235, pp. 1490--1501, 2011.
%
% \bibitem{JZ13}
% {\sc X.\ Jiang and Y.\ Zhang},
% {\em A smoothing-type algorithm for absolute value equations},
% Journal of Industrial and Management Optimization, vol. 9, pp. 789--798, 2013.
%
% \bibitem{KM12}
% {\sc S.\ Ketabchi and H.\ Moosaei},
% {\em Minimum norm solution to the absolute value equation in the convex case},
% Journal of Optimization Theory and Applications, vol. 154, pp. 1080--1087, 2012.

 \bibitem{KR92}
 {\sc J.\ Kreimer and R.Y.\ Rubinstein},
 {\em Nondifferentiable optimization via smooth approximation: General analytical approach},
 Annals of Operations Research, vol. 39, pp. 97--119, 1992.

 \bibitem{LQ99}
 {\sc L.-Z.\ Liao and H.-D.\ Qi},
 {\em A neural network for the linear complementarity problem},
 Mathematical and Computer Modelling, vol. 29, pp. 9--18, 1999.

 \bibitem{LQQ01}
 {\sc L.\ Liao, H.-D.\ Qi, and L.\ Qi},
 {\em Solving nonlinear complementarity problem with neural networks: a reformulation method approach},
 Journal of Computational and Applied Mathematics, vol. 131, pp. 343--359, 2001.


\bibitem{M1985}
 {\sc O.L. Mangasarian},
  {\em A CONDITION NUMBER FOR DIFFERENTIABLE CONVEX INEQUALITIES},
   MATHEMATICS OF OPERATIONS RESEARCH, vol.10(2),pp. 175–179, 1985.


\bibitem{M1995}
 {\sc O.L. Mangasarian},
  {\em A condition number for differentiable convex inequalities},
   Mathematics of Operations Research, vol.10(2),pp. 175-179, 1985.
 %\bibitem{Man07-1}
% {\sc O.L.\ Mangasarian},
% {\em Absolute value programming},
% Computational Optimization and Applications, vol. 36, pp. 43--53, 2007.
%
% \bibitem{Man07-2}
% {\sc O.L.\ Mangasarian},
% {\em Absolute value equation solution via concave minimization},
% Optimization Letters, vol. 1, pp. 3--5, 2007.
%
% \bibitem{Man09}
% {\sc O.L.\ Mangasarian},
% {\em A generalized Newton method for absolute value equations},
% Optimization Letters, vol. 3, pp. 101--108, 2009.
%
% \bibitem{Man12}
% {\sc O.L.\ Mangasarian},
% {\em Primal-dual bilinear programming solution of the absolute value equation},
% Optimization Letters, vol. 6, pp. 1527--1533, 2012.
%
% \bibitem{Man13}
% {\sc O.L.\ Mangasarian},
% {\em Absolute value equation solution via dual complementarity},
% Optimization Letters, vol. 7, pp. 625--630, 2013.
%
% \bibitem{MHC17}
% {\sc X.-H.\ Miao, W.-M.\ Hsu, and J.-S.\ Chen},
% {\em The solvabilities of three optimization problems associated with second-order cone},
% submitted manuscript, 2018.
%
% \bibitem{MYSC17}
% {\sc X.-H.\ Miao, J.-T.\ Yang, B.\ Saheya, and J.-S.\ Chen},
% {\em A smoothing Newton method for absolute value equation associated with second-order cone},
%  Applied Numerical Mathematics, vol. 120, October, pp. 82--96, 2017.
%
% \bibitem{MM06}
% {\sc O.L.\ Mangasarian and R.R.\ Meyer},
% {\em Absolute value equation},
% Linear Algebra and Its Applications, vol. 419, pp. 359--367, 2006.

 \bibitem{Mor65}
 {\sc J.J.\ Moreau},
 {\em Proximit$\acute{e}$ et dualit$\acute{e}$ dans un espace Hilbertien},
 Bulletin de la Soci$\acute{e}$t$\acute{e}$ Math$\acute{e}$matique de France,
 vol. 93,  pp. 273-299, 1965.

 \bibitem{Nesterov05}
 {\sc Y.\ Nesterov},
 {\em Smooth minimization of non-smooth functions},
 Mathematical Programming, vol. 103(1), pp. 127--152, 2005.

 \bibitem{PZ1995}
 {\sc Pinar M. C., Zenios S. A.},
 {\em On smoothing exact penalty functions for convex constrained optimization},
 SIAM Journal on Optimization, 4(3): 486-511, 1994.

% \bibitem{Pro09}
% {\sc O.A.\ Prokopyev},
% {\em On equivalent reformulations for absolute value equations},
% Computational Optimization and Applications, vol. 44, pp. 363--372, 2009.

 \bibitem{Qi93}
 {\sc L.\ Qi},
 {\em Convergence analysis of some algorithms for solving nonsmooth equations},
 Mathematics of Operations Research, vol. 18, pp. 227--244, 1993.

 \bibitem{QSZ00}
 {\sc L.\ Qi, D.\ Sun, and G.-L.\ Zhou},
 {\em A new look at smoothing Newton methods for nonlinear complementarity problems
 and box constrained variational inequality problems},
 Mathematical Programming, vol. 87, pp. 1--35, 2000.

 \bibitem{QD02}
 {\sc L.\ Qi and D.\ Sun},
 {\em Smoothing functions and smoothing Newton method for complementarity and variational
 inequality problems},
 Journal Of Optimization Theory And Applications, vol. 113, pp. 121--147, 2002.

 %\bibitem{Roh04}
% {\sc J.\ Rohn},
% {\em A theorem of the alternatives for the equation $Ax+B|x|=b$},
% Linear and Multilinear Algebra, vol. 52, pp. 421--426, 2004.
%
% \bibitem{Roh06}
% {\sc J.\ Rohn},
% {\em Solvability of systems of interval linear equations and inequalities},
% in Linear Optimization Problems with Inexact Data edited by. M. Fiedler, J. Nedoma,
% J. Ramik, J. Rohn and K. Zimmermann, Springer, pp. 35--77, 2006.
%
% \bibitem{Roh09}
% {\sc J.\ Rohn},
% {\em An algorithm for solving the absolute value equation},
% Eletronic Journal of Linear Algebra, vol. 18, pp. 589--599, 2009.

 \bibitem{SYC16}
 {\sc B.\ Saheya, C.-H.\ Yu, and J.-S.\ Chen},
 {\em Numerical comparisons based on four smoothing functions for absolute value equation},
 Journal of Applied Mathematics and Computing, vol. 56, pp. 131--149, 2018.

 \bibitem{SCK12}
 {\sc J.-H.\ Sun, J.-S.\ Chen, and C.-H.\ Ko},
 {\em Neural networks for solving second-order cone constrained variational inequality problem},
  Computational Optimization and Applications, vol. 51, pp. 623--648, 2012.

 \bibitem{TH86}
 {\sc D.W.\ Tank and J.J.\ Hopfield},
 {\em  Simple neural optimization networks: an A/D converter, signal decision circuit,
 and a linear programming circuit},
 IEEE Transactions on Circuits and Systems. vol. 33, pp. 533--541, 1986.

 \bibitem{VOY15}
 {\sc S.\ Voronin, G.\ Ozkaya, and D.\ Yoshida},
 {\em Convolution based smooth approximations to the absolute value function with application
 to non-smooth regularization},
 arXiv:1408.6795v2 [math.NA] 1 Jul 2015.

 %\bibitem{WYG15}
% {\sc F.\ Wang, Z.\ Yu, and C.\ Gao},
% {\em A smoothing neural network algorithm for absolute value equations},
% Engineering, vol. 7, pp. 567--576, 2015.
%
% \bibitem{YT12}
%  {\sc Yong, L.Q., Liu, S.Y. and Tuo, S.H.},
% {\em Transformation of the Linear Complementarity Problem and the Absolute Value Equation},
%  Journal of Jilin University (Science Edition), vol. 4, pp. 638-686, 2014.
%
% \bibitem{YF14}
% {\sc S.\ Yamanaka and M.\ Fukushima},
% {\em A brancd and bound method for the absolute value programs},
% Optimization, vol. 63, pp. 305--319, 2014.
%
% \bibitem{ZW09}
% {\sc C.\ Zhang and Q.\ Wei},
% {\em Global and finite convergence of a generalized Newton method for absolute value equations},
% Journal of Optimization Theory and Applications, vol. 143, pp. 391--403, 2009.

 \bibitem{ZUH95}
 {\sc S.H.\ Zak, V.\ Upatising, and S.\ Hui},
 {\em Solving linear programming problems with neural networks: a comparative study},
 IEEE Trans Neural Networks, vol. 6, pp. 94--104, 1995.


 %\bibitem{FLT02}
% {\sc M.\ Fukushima, Z.Q.\ Luo, and P.\ Tseng},
% {\em Smoothing functions for second-order cone complementarity problems},
% SIAM Journal on Optimization, vol. 12, pp. 436-460, 2002.
%
%
% \bibitem{HYF05}
% {\sc S.\ Hayashi, N.\ Yamashita, and M.\ Fukushima},
% {\em A combined smoothing and regularization method for monotone second-order cone
% complementarity problems},
% SIAM Journal on Optimization, vol. 15, pp. 593--615, 2005.

\bibitem{HZW08}
 {\sc Z.H.\ Huang, Y.\ Zhang, W.\ Wu},
 {\em A smoothing-type algorithm for solving a system of inequalities},
 Journal of Computational and Applied Mathematics, vol. 220, pp. 355--363, 2008.


 \bibitem{ZH09}
 {\sc Y.\ Zhang and Z.-H.\ Huang},
 {\em A nonmonotone smoothing-type algorithm for solving a system of equalities and inequalities},
 Journal of Computational and Applied Mathematics, vol. 233, pp. 2312--2321, 2010.

 \bibitem{saheya2019}
  {\sc Saheya, B., Nguyen, C.T., Chen, J.-S.},
   {\em Neural network based on systematically generated smoothing functions for absolute value equation},
    J. Appl. Math. Comput., 2019

 \bibitem{GuerraVazquez2001}
 {\sc Guerra Vazquez, F., Gnzel, H., Jongen, H.T.},
 {\em  On Logarithmic Smoothing of the Maximum Function},
 Annals of Operations Research,vol.101,pp. 209-220, 2001.

 \bibitem{Zang1980}
 {\sc Israel Zang},
 {\em A smoothing-out technique for min-max optimization},
  Mathematical Programming, 19:61-77, 1980.

 \end{thebibliography}


 \end{document}









